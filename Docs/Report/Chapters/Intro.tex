\chapter{Introduction}\label{C:Introduction}
The aim of this project is to develop a Java Virtual Machine and design Java specific API to provide a easy mechanism to use microcontroller resources using similar concept used in JavaCard for SmartCards.

The main idea is to implement a JVM which could be executed in microcontrollers based on 1 kB RAM or more. This JVM is developed in C and provides a Java API to get access to microcontroller resources (ports, timers, serial port, etc.). It allows to implement microcontrollers Java applications and that means support to object oriented programming and compatible with any microcontroller running this JVM on it, i.e. one Java application that will work in all microcontrollers.

This JVM should be a reduced version of standard JVM in order to successfully execute it on low and mid range microncontroller with limited resources. This project is based on JavaCard for SmartCards, for that reason, an introduction and comparison with this technology and how it is currently used on real life using high range microcontrollers will be analyzed. JVM will be initially developed to PIC18F4520 and PIC16F87X and adaptation to other devices is out of the scope an open for future works.

In order to check the behavior of Java for microcontrollers designed for this project in accuracy environments, during second part of this report is going to be analyzed a direct interface circuit for resistive sensors by developing a native application in C based on \cite{Art:Accuracy} methods. The same application is developed in Java by using API and JVM described and designed in this report. Then, final goal is to make a comparison, in terms accuracy, between applications based on this Java and native application. Regarding accuracy, resistive capture method by microcontroller PIC18F4520 will be analized.

First chapter exposes the motivation of this project and why it is interesting to introduce a programming language like Java in a microcnotrollers world. Existing preceding implementations of VM\nomenclature{VM}{Virtual Machine} are compared and why it is required a new one. Second chapter explains the complete design of JVM developed and how it works. During this chapter, the process since a Java application is created until it starts to run is explained in details. Then, third chapter introduces the theory related to direct interface circuits for resistive sensors and different implementations of an application developed for this, one in C and other one in Java, are compared. Finally, report concludes with describing conclusions extracted from results obtained.
